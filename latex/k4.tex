\chapter{Die ökonomische Effizienz des elektronischen Marktes für Fußballwetten in der Saison 2013/14}

In den vergangenen Kapiteln wurde die Frage nach der sozialen Verträglichkeit des Sportwettenmarktes geklärt und in die Theorie der Wettquoten eingeführt. Im Folgenden wird sich dieses Kapitel mit der ökonomischen Effizienz befassen. Insbesondere konzentriert sich dieser Teil der Arbeit auf die Effizienz des elektronischen Marktes für Fußballwetten in der Saison 2013/14. 

In Kapitel 3 wurde bereits auf die Begriffe \textit{Favourite-Longshot Bias} und Arbitrage eingegangen. In diesem Kapitel wird an diese Begriffe angeschlossen und diese näher untersucht. Dazu muss jedoch der Begriff der ökonomischen Effizienz im Bezug auf den Sportwettenmarkt definiert werden. Der Wettmarkt lässt sich als Informationsmarkt charakterisieren. Das heißt bei den Marktteilnehmern handelt es sich um Informationsanbieter (Buchmacher) und Informationsnachfrager (Spieler), welche einen Informationsaustausch durchführen. Diese Charakteristik wurde erstmals von \citet{fama1970efficient} für Aktienmärkte eingeführt und hat sich seitdem bei der Analyse von Wettmärkten ebenfalls bewährt. Auf dem Aktienmarkt findet durch den Kauf beziehungsweise Verkauf von Aktien ein Informationsaustausch statt. Beim Handeln von Aktien werden sich die Marktteilnehmer (vor allem jedoch die Parteien, welche in dem Handel involviert sind) die Frage stellen, auf Grundlage welcher Information handelt mein jeweiliger Gegenüber. Welche Motive veranlassen diesen zu einem Kauf oder Verkauf der gewünschten Aktie. Dieses Prinzip lässt sich äquivalent auf den Wettmarkt für Fußballwetten übertragen. Als anschauliches Beispiel dient dazu eine Wette mit Favoritentendenz, das heißt der Buchmacher ordnet einer Mannschaft die Favoritenrolle zu. Aus dem Kapitel der Wettquotentheorie folgt, dass der Buchmacher seine implizite Wahrscheinlichkeit für den Sieg der favorisierten Mannschaft groß wählen wird. Daraus ergibt sich sofort, dass die Quote für einen Sieg des Favoriten dementsprechend klein sein und nahe bei Eins liegen wird. Kommt ein Wettender durch eigene Einschätzungen nun zu der Auffassung, dass die Gewinnwahrscheinlichkeit des Favoriten überbewertet ist, wird sich dieser die Frage stellen, welche Informationen besitzt der Buchmacher, die ihm eventuell verborgen geblieben sind. Diese Gedankengänge funktionieren natürlich auch in entgegengesetzter Art und Weise. Stellt der Buchmacher fest, dass die Mehrheit der Spieler auf den Außenseiter setzen, hat dieser zwei Möglichkeiten: Zum einen wird er sich fragen können, ob er einen Fehler bei der Einschätzung der Gewinnwahrscheinlichkeit des Favoriten gemacht hat. Zum zweiten kann er jedoch aus Kalkül gehandelt haben um durch die Unterbewertung des Außenseiters die hohe Quoten für dessen Sieg als Lockvogel zu nutzen um so seinen Erwartungsgewinn zu maximieren. Auf diese Vorgehensweise eines Buchmachers wurde bereits im Abschnitt über \textit{risikominimale Wettquoten} und bei der Interpretation des FLB eingegangen.

Damit ist die Charakterisierung des Wettmarktes als Informationsmarktes erklärt. \citet{fama1970efficient} definiert in seiner Arbeit einen effizienten Informationsmarkt dadurch, dass die Preise alle Informationen beinhalten müssen. \citeauthor{fama1970efficient} unterschied dabei zwischen schwacher und starker Markteffizienz. Bei einem Test auf \textit{schwache Markteffizienz} sollen nur Preise vergangener Perioden betrachtet werden. Für den Test auf \textit{starke Markteffizienz} gehen alle Informationen in die Analyse ein. Insbesondere auch Informationen, welche nur speziellen Gruppen vorliegen und den restlichen Marktteilnehmern verborgen sind. \citet{thaler1988parimutuel} vertraten als Erste die Auffassung, dass Wettmärkte im Hinblick auf Effizienztest die besseren Untersuchungsgegenstände seien. Der Vorteil von Wettmärkten gegenüber anderen Finanzmärkten (etwa dem Aktienmarkt) ist, dass Quoten zu einem bestimmten Zeitpunkt ihren Wert annehmen. Das heißt, sie besitzen eine genau definierte Lebensdauer, dessen Ende durch den Abpfiff des Fußballspiels festgelegt ist. Durch diese Eigenschaft vereinfacht sich das Testen auf Markteffizienz \citep[S. 427]{vlastakis2009efficient}. \citet[S. 163]{thaler1988parimutuel} fassten die Idee von \citeauthor{fama1970efficient} auf und formulierten die Effizienzbedingungen erstmals für den Sportwettenmarkt für Pferdewetten:

\begin{definition}
\begin{itemize}
\item[\textbf{(a)}] \textbf{Schwache Markteffizienz:} Es dürfen keine Wetten mit positiven Erwartungsgewinne für die Wettenden existieren.
\item[\textbf{(b)}] \textbf{Starke Markteffizienz:} Alle Wetten sollen den selben Erwartungsgewinn in Höhe der (relativen) Gewinnmarge haben.
\end{itemize}
\label{def_markteffizienz}
\end{definition}


Diese Definitionen haben sich bis heute bei der Analyse von (Sport-)Wettmärkten durchgesetzt. Demzufolge sind sie auch prägend für die Untersuchung von Fußballwettquoten. Bei der Einführung in die Wettquotentheorie in Kapitel 3 wurde auf ein Zitat von \citep[S. 1355]{kuypers2000information} verwiesen, welches klarstellte, dass die Menge der privaten Informationen im modernen Fußball kaum existent ist. Insbesondere gilt das für die oberen Ligen in den wichtigsten Fußballnationen Europas: Die 1. und 2. deutsche Bundesliga, Spaniens \textit{Primera \& Segunda División}, die \textit{Premier League} sowie \textit{Football League Championship} in England, die italienische \textit{Serie A} und in Frankreich die \textit{Le Championat}. Die allgemeine Beliebtheit des Fußballsports, das öffentliche sowie das Medieninteresse ist in diesen Verbänden derart stark ausgeprägt, dass die Vorstellung es würden private Informationen existieren, welche nur den Buchmachern vorliegen, äußerst unrealistisch erscheint.\footnote{Der Vollständigkeit halber sollte erwähnt werden, dass dennoch vereinzelt Berichte von Wettskandalen in den populärsten Ligen zu finden sind. Der Fall des deutschen Schiedsrichters Robert Hoyzer aus dem Jahr 2005 ist wohl einer der populärsten Fälle.} Auf Grund der Prämisse von \citeauthor{kuypers2000information} konzentriert sich diese Arbeit im weiteren Verlauf auf den Nachweis der \textit{schwachen Markteffizienz} des Fußballwettmarktes durch die Untersuchung von Wettquoten der Saison 2013/14. In seiner Hauptaufgabe wird sich dieses Kapitels mit der Akkurarität der impliziten zu den wahren Wahrscheinlichkeiten befassen und anstreben die Frage nach lukrativen Wettstrategien zu klären. Beispiele für solche Strategien sind: Nur Heim-, Auswärtssieg oder Unentschieden zu wetten; immer auf die Favoriten oder Außenseiter zu setzen; alle Ergebnisse zu tippen. Vor allem die zuletzt genannte Strategie spielt bei der Suche von Arbitragemöglichkeiten eine Hauptrolle. 

Die nachfolgenden Ergebnisse werden in Verbindung mit anderen Arbeiten auf dem Gebiet der Sportwetten gebracht, speziell mit \citet{kossmeier2008efficiency} und \citet{vlastakis2009efficient}. Daher sei an dieser Stelle noch auf die verwendeten Datensätze der anderen Autoren verwiesen. \citeauthor{kossmeier2008efficiency} untersuchten für ihre Arbeit 1800 Spiele aus den größten europäischen Ligen (Österreich, England, Frankreich, Deutschland, Italien und Spanien) aus der Saison 2000/01. Insgesamt verglichen sie die Quoten von zwölf Buchmachern. \citeauthor{vlastakis2009efficient} werteten die Quoten von fünf Online-Buchmachern und einem Offline-Anbieter\footnote{Bei diesem Anbieter handelt es sich um einen Buchmacher vergleichbar mit dem deutschen Oddset-Angebot des DTLB.} aus Griechenland aus. Ihr Datensatz umfasste  12,841 Spiele aus den Jahren 2002-2004. Für ihre Untersuchung auf lukrative Wettstrategien reduzierte sich ihr Datensatz auf vier Buchmacher und 1486 Spiele.


\section{Diskriptive Statistik \& die ökonomische Interpretation des \textit{Takes}}


\begin{table}
\centering
	\begin{tabular}{lcccr}
	\toprule
	\textbf{Liga} & \textbf{Heimsieg} & \textbf{Unentschieden} & \textbf{Auswärtssieg} & \textbf{Summe} \\
	\midrule
	\textit{1. Bundesliga} & 145 & 63 & 96 & 304 \\
	\textit{2. Bundesliga} & 125 & 87 & 92 & 304 \\
	\textit{Le Championnat} & 168 & 108 & 103 & 379 \\
	\textit{Primera Division} & 178 & 86 & 114 & 378 \\
	\textit{Premier League} & 179 & 78 & 123 & 380 \\
	\textit{Seria A} & 180 & 90 & 109 & 379\\
	\midrule
	\textbf{gesamt:} & 975 & 512 & 637 & 2124\\
	\textbf{relativ:} & 0.4590 & 0.2411 & 0.2999 & 1.000\\
	\toprule
	\end{tabular}

\caption[Spielausgänge in ausgewählten Ligen Saison 2013/14]{Deskriptive Statistik der Spielausgänge in den untersuchten Ligen Saison 2013/14 \\ Quelle: Eigene Darstellung u. Berechnung}
\label{tab:statisktik:spielausgänge}
\end{table}

Für die eigene Datenanalyse dieser Arbeit wurden die Spiele aus den sechs größten Fußballligen in Europa der Saison 2013/14 untersucht. Die untersuchten Ligen sind in Tabelle~\ref{tab:statisktik:spielausgänge} aufgelistet. Es handelt sich um die \textit{1. \& 2. Bundesliga} (Deutschland), \textit{Le Championat} (Frankreich), \textit{Primera Division} (Spanien), \textit{Premier League} (England) und \textit{Seria A} (Italien). Auf Grund fehlender Quoten einzelner Buchmacher wurden aus dem Datensatz 8 Spiele gestrichen. Demnach umfasst der Datensatz 2124 Begegnungen in sechs Ligen. Aus Tabelle~\ref{tab:statisktik:spielausgänge} geht außerdem hervor, dass in der Saison 2013/14 die Spiele durchschnittlich zwischen 41\% (2. Bundesliga) und 48\% (\textit{1. Bundesliga}) beziehungsweise 27\% (\textit{Le Championat}) und 32\% (\textit{Premier League}) mit einem Sieg für die Heim- beziehungsweise Auswärtsmannschaft endeten. Der Anteil der Partien, die unentschieden endeten beläuft sich auf 21\% (\textit{Premier League}) bis 29\% (\textit{2. Bundesliga}). Im Durchschnitt über alle Ligen endeten 45.9 von 100 Spielen mit einem Sieg der Heimmannschaft. Für die Spielausgänge Unentschieden beziehungsweise Auswärtssieg ergeben sich 24.11 beziehungsweise 29.99 auf 100 Spielen.

Der Vergleich der Quoten findet auf Basis von acht Online-Buchmachern statt: \textit{Bet365} (Firmensitz England), \textit{Bwin} (Gribaltar), \textit{Interwetten} (Malta), \textit{Ladbrokes} (England), \textit{Pinnacle Sports} (Curaçao), \textit{William Hill} (England), \textit{Stan James} (Gibraltar), \textit{Victor Chandler} (Gibraltar). Zu Gunsten der Übersichtlichkeit werden diese im Folgenden als \textit{Buchmacher B365, BW, IW, LB, PS, WH, SJ, VC} bezeichnet.

\begin{table}
\centering
	\begin{tabular}{lccccrr}
	\toprule
	  & \multicolumn{4}{c}{\textbf{Take $ T $}} & \multicolumn{2}{c}{\textbf{Gewinnmarge}}\\
	\midrule
	\textbf{Anbieter} & \textbf{Min.} & \textbf{Max.} & \textbf{Mittel} & \textbf{Stabw} & \textbf{absolut} $ \tau $ & \textbf{relativ} $ t $\\
	\midrule
	\textit{Buchmacher B365} & 1.0184 & 1.0941 & 1.0574 & 0.0153 & 5.74\% & 5.41\% \\
	\textit{Buchmacher BW} & 1.0350 & 1.1538 & 1.0680 & 0.0070 & 6.80\% & 6.37\% \\
	\textit{Buchmacher IW} & 1.0538 & 1.0965 & 1.0806 & 0.0020 & 8.06\% & 7.46\% \\
	\textit{Buchmacher LB} & 1.0332 & 1.1188 & 1.0671 & 0.0096 & 6.71\% & 6.28\% \\
	\textit{Buchmacher PS} & 1.0158 & 1.0399 & 1.0223 & 0.0035 & 2.23\% & 2.18\% \\
	\textit{Buchmacher WH} & 1.0158 & 1.1210 & 1.0377 & 0.0270 & 3.77\% & 3.57\% \\
	\textit{Buchmacher SJ} & 1.0215 & 1.1044 & 1.0697 & 0.0156 & 6.97\% & 6.50\% \\
	\textit{Buchmacher VC} & 1.0124 & 1.0906 & 1.0324 & 0.0085 & 3.24\% & 3.13\% \\
	\toprule
	\end{tabular}
	
\caption[Gewinnmarge je Buchmacher über alle Spiele der Saison 2013/14]{Deskriptive Statistik der Gewinnmarge je Buchmacher über alle Spiele der Saison 2013/14 \\ Quelle: Eigene Darstellung u. Berechnung}
\label{tab:statistik:gewinnmarge}
\end{table}

In Tabelle~\ref{tab:statistik:gewinnmarge} sind die verschiedenen Takes und Gewinnmargen der einzelnen Buchmacher über alle Spiele aufgelistet. Im Durchschnitt über alle Spiele und alle Buchmacher beträgt die absolute beziehungsweise relative Gewinnmarge 5.44\% beziehungsweise 5.14\%. Bei genauerer Betrachtung fällt die große Varianz zwischen den einzelnen Gewinnmargen der Buchmacher auf. Die absolute Gewinnmarge $ \tau $ reicht von 2.23\% (\textit{Pinnacle Sports}) bis 8.06\% (\textit{Interwetten}). Für den Vergleich mit Daten aus zugehöriger Literatur zum Thema Sportwetten ist vor allem die relative Gewinnmarge $ t $ relevant. Sie ist äquivalent zum Erwartungsgewinn des Buchmachers. Der Zusammenhang zwischen Take und relativer Gewinnmarge wurde bereits in Kapitel 3 eingeführt. Er ergibt sich aus Satz~\ref{satz_implizite_Wkeit_auszahlung} als $ t = 1-\tfrac{1}{T} $. Die relative Gewinnmarge schwankt zwischen 2.18\% und 7.46\%. Die Standardabweichungen größer als Null sowie die unterschiedlichen minimalen beziehungsweise maximalen Takes machen deutlich, dass dieser über die Saison nicht konstant ist. Diese Ergebnisse unterscheiden sich gravierend von denen aus benachbarten Arbeiten. In einer Effizienzuntersuchung der Saisons 1993/94/95 durch \citet{kuypers2000information} stellte dieser im Mittel eine absolute Gewinnmarge von 11.5\% fest. In einer aktuelleren Arbeit von \citet{kossmeier2008efficiency} wurden Wettquoten der Saison 2000/01 untersucht. Der Durchschnitt der dort ermittelten relativen Gewinnmarge liegt bei 14.9\%. \citet{vlastakis2009efficient} untersuchten den elektronischen Markt für Fußballwetten an Hand von Daten aus den Saisons 2002/03/04 und die festgestellten relativen Gewinnmargen von vier Online-Buchmachern liegt zwischen 11.68\% und 16.28\%. \citet{barth2012oekonomie} untersuchte den Wettmarkt in den Saisons 2005/06/07/08. In seinem Datensatz lassen sich ebenfalls drei Online-Buchmacher wiederfinden. Die relative Gewinnmarge über jede Saison und jeden Wettanbieter liegt dort zwischen 7.5\% und 11.5\%.

\begin{figure}
 \centering
 	\includegraphics[width=0.8\textwidth]{bilder/gewinnmarge3.eps}
\caption[Entwicklung der Gewinnmarge in den Jahren 1993 bis 2013/14]{Entwicklung der durchschnittlichen relativen Gewinnmargen in den Jahren 1993 bis 2013/14. Die Daten stammen aus ausgewählten Arbeiten zum Thema Sportwetten. Die Autoren sind in der Legende gelistet. \\ Quelle: Eigene Darstellung}
\label{fig:entwicklung_gewinnmarge}
\end{figure}

In Abbildung~\ref{fig:entwicklung_gewinnmarge} ist die Entwicklung der relativen Gewinnmarge von 1993 bis 2014 dargestellt. Auf Grund fehlender Daten für den Zeitraum 2009 bis einschließlich 2012 lassen sich daher nur Mutmaßungen über die Entwicklung der Gewinnmarge anstellen. Lässt man die Untersuchung von \citet{kuypers2000information} außen vor, lässt sich ab dem Jahr 2000 ein klarer Abwärtstrend erkennen. Die Gewinnmarge lässt sich im ökonomischen Sinne dahingehend interpretieren, dass es sich um den ''internen Verrechnungspreis'' der Wettangebote handelt. Die Gewinnmarge ist ausschlaggebend für den Buchmachervorteil. Im Glücksspiel gilt trivialer Weise der negative Zusammenhang zwischen dem Buchmacher-  und dem Spielervorteil. Folglich bedeutet das, je kleiner die Gewinnmarge des Buchmachers desto größer der Spielervorteil. Daher konkurrieren Buchmacher untereinander mit der Höhe ihrer jeweiligen Gewinnmargen. Denn der Vorteil für die Kunden (Spieler), eine Wette bei einem Anbieter mit geringer Gewinnmarge abzuschließen, besteht in höheren Wettquoten und somit größeren Gewinnen. Diese Tatsache ergibt sich mathematisch sofort aus dem Zusammenhang zwischen Quote und impliziter Wahrscheinlichkeit (vgl. Kapitel 3 Abschn. 2). Die in den letzten Jahren festgestellte Abnahme der Gewinnmargen lässt auf die Zunahme des Wettbewerbs unter den Wettanbietern schließen. Zusätzlich ist der Quotenvergleich per Internet heutzutage sehr leicht, was den Wettbewerb zwischen den Anbieter weiter schürt. Als Gründe für einen Wettbewerbszuwachs können die Deregulierung des Glücksspielmarkts in Europa sowie die wachsende Beliebtheit im Online-Glücksspielgeschäft angeführt werden (vgl. Kapitel 2). Die weitere Entwicklung der Gewinnmarge in den kommenden Jahren bleibt eine interessante Untersuchungsaufgabe.


\begin{table}
\centering
	\begin{tabular}{lccc}
	\toprule
	\textbf{Anbieter} & $ \phi_H $ & $ \phi_U $ & $ \phi_A $ \\
	\midrule
	\textit{Buchmacher B365} & 0.4520 (0.1757) & 0.2552 (0.0478) & 0.2928 (0.1568) \\
	\textit{Buchmacher BW} & 0.4509 (0.1729) & 0.2543 (0.0472) & 0.2948 (0.1550) \\
	\textit{Buchmacher IW} & 0.4486 (0.1651) & 0.2562 (0.0420) & 0.2952 (0.1476) \\
	\textit{Buchmacher LB} & 0.4497 (0.1694) & 0.2556 (0.0446) & 0.2947 (0.1519) \\
	\textit{Buchmacher PS} & 0.4556 (0.1694) & 0.2551 (0.0513) & 0.2893 (0.1620) \\
	\textit{Buchmacher WH} & 0.4526 (0.1796) & 0.2573 (0.0502) & 0.2901 (0.1593) \\
	\textit{Buchmacher SJ} & 0.4499 (0.1725) & 0.2566 (0.0459) & 0.2935 (0.1544) \\
	\textit{Buchmacher VC} & 0.4527 (0.1783) & 0.2560 (0.0494) & 0.2913 (0.1584) \\
	\midrule
	\textit{Mittel:} & 0.4515 & 0.2558 & 0.2927 \\
	\textit{Ex-post:} & 0.4590 & 0.2411 & 0.2999 \\
	\toprule
	\end{tabular}
	
\caption[Vergleich der impliziten Wahrscheinlichkeiten Saison 2013/14]{Deskriptive Statistik der impliziten Wahrscheinlichkeiten der untersuchten Buchmacher über alle Spiele der Saison 2013/14. Die Werte in Klammern beziehen sich auf die Standardabweichung der zugehörigen impliziten Wahrscheinlichkeit bzgl. der Spielausgänge Heimsieg \textit{(H)}, Unentschieden \textit{(U)}, Auswärtssieg \textit{(A)}. \\ Quelle: Eigene Darstellung u. Berechnung}
\label{tab:statistik:impliziteWKeit}
\end{table}

In Kapitel 3 wurde in Beispiel~\ref{bsp_implizite_Wkeit} sowie darüber hinaus von den impliziten Wahrscheinlichkeiten der Quoten gesprochen. Die impliziten Wahrscheinlichkeiten waren die subjektiven Einschätzungen der Buchmacher, vermindert durch den Take $ T $. In gängiger Literatur zum Thema Sportwetten ist die Annahme üblich, dass die Gewinnmarge des Buchmachers und folglich der Take zu gleichen Teilen auf die subjektive Einschätzung aufgeschlagen wird. Diese Annahme bleibt hier bestehen. Mit den drei Spielausgängen $ k=H,U,A $ ergibt sich ein System von vier Gleichungen mit vier Unbekannten: \begin{align*}
\phi_k = \frac{1}{T \, q_k} \quad \mbox{ ,für } k=H,U,A \\
\sum_{k}^{}{\phi_k} = 1
\end{align*}
Durch die Anwendung dieses Systems, lassen sich die impliziten Wahrscheinlichkeiten der Buchmacher bestimmen. Die durchschnittlichen impliziten Wahrscheinlichkeiten der untersuchten Buchmacher ist in Tabelle~\ref{tab:statistik:impliziteWKeit} dargestellt. Die Berechnung der Durchschnittsquoten für die Saison 2013/14 ergibt ein Quotentripel mit $ q_k = 2.64/3.94/4.75 $ für $ k=H,U,A $. In Tabelle~\ref{tab:statistik:impliziteWKeit} lässt sich außerdem erkennen, dass die Buchmacher die Eintrittswahrscheinlichkeit eines Heimsiegs zwischen 44.86\% und 45.20\% bewerten. Die Wahrscheinlichkeiten für ein Unentschieden liegen zwischen 25.43\% und 25.73\%. Hier weisen die Wahrscheinlichkeiten eine sehr geringe Varianz auf, was sich ebenfalls an den geringen Standardabweichungen von maximal 5.13\% erkennen lässt. Die Eintrittswahrscheinlichkeit eines Auswärtssiegs sind mit 28.93\% bis 29.52\% bemessen. Es fällt auf, dass alle Buchmacher nahezu eine ähnliche Einschätzung der Eintrittswahrscheinlichkeit bezüglich der Ereignisse $ k=H,U,A $ besitzen. Dieser Umstand wird durch die fast gleichen Standardabweichungen aller Buchmacher unterstützt. Diese Tatsache überrascht, denn obwohl die Einschätzungen der Buchmacher übereinstimmen unterscheiden sie sich enorm in der Höhe ihrer Gewinnmargen (vgl. Tabelle~\ref{tab:statistik:gewinnmarge}). Damit ergeben sich für den Wettenden unterschiedlich hohe Wettquoten und folglich voneinander abweichend hohe Gewinne.

Zusätzlich lässt sich aus Tabelle~\ref{tab:statistik:impliziteWKeit} keine nennenswerte Unter- oder Überbewertung der impliziten Wahrscheinlichkeiten im Verhältnis zu den wahren Eintrittswahrscheinlichkeiten  erkennen (in Tabelle~\ref{tab:statistik:impliziteWKeit} in der Zeile \textit{Ex-post} angegeben). Die Überbewertung des Spielausgangs Unentschieden weicht im Mittel um 1.47 Prozentpunkte ab. Die Unterbewertung der Wahrscheinlichkeit eines Heimsiegs beziehungsweise Auswärtssiegs beträgt im Mittel 0.75 Prozentpunkte beziehungsweise 0.72 Prozentpunkte. Die Arbeit von \citet[S. 450]{kossmeier2008efficiency} stellte einen sogenannten \textit{Home-bias} (dt. Heimverzerrung) fest. Dort betrug die mittlere Abweichung jedoch mehr als 4 Prozentpunkte. Die durchschnittliche Bewertung der Buchmacher eines Heimsiegs belief sich in der Saison 2000/01 auf 46.30\%. In Wahrheit endeten jedoch 50.6 von 100 Spielen mit einem Sieg der Heimmannschaft. Auf Grund der geringen Abweichung in dem hier untersuchten Datensatz kann in keinster Weise von dem Vorhandensein einer (Heim-, Auswärts-, Unentschieden-)Verzerrung gesprochen werden.

Außerdem ist auffällig, wie exakt die Buchmacher die Eintritte der Fußballspiele vorhersagen können. Im Rahmen der Effizienzmarkthypothese wird diese bestätigt, da es sich durch diese exakte Vorhersage nach Satz~\ref{satz_implizite_Wkeit_auszahlung} folglich um \textit{markteffiziente Wettquoten} handelt. Die nachgewiesenen Ergebnisse sind daher konform mit Definition~\ref{def_markteffizienz} (a) und bestätigen so die Effizienzmarkthypothese. Diese Tatsache unterstreicht insbesondere die in Kapitel 3 zitierte These von \citet[S. 243]{levitt2004gambling}, welche besagt, Buchmacher sind in jedem Fall die ''besten Schätzer'' bezüglich der Ausgänge eines Sportereignisses.


\section{Arbitrageanalyse der Saison 2013/14}
Im diesem Abschnitt wird die Effizienz des elektronischen Marktes für Fußballwetten vor dem Hintergrund von Arbitrage untersucht. In Kapitel 3 wurde diesem Teilgebiet der Effizienzanalyse bereits ein Abschnitt gewidmet. Dort wurde insbesondere die Existenzbedingung für Arbitrage hergeleitet. Ein Algorithmus, welcher das Auswerten von Wettquoten für Arbitragemöglichkeiten untersucht ist in Algorithmus~\ref{alg:arbitrage} dargestellt. 


\begin{algorithm}
\caption{Arbitrage}\label{alg:arbitrage}
\begin{algorithmic}[1]
\Require Liste mit Quoten $q_{jk}$ für alle Buchmacher $j \in J=\{1,\dots,n\}$ mit $ |J| \geq 2 $
\Ensure Indizes $j$ bezüglich der Ausgänge $k$
\Procedure{Arbitrage}{$q_{jk}$}
\ForAll{$k=H,U,A$}
	\For{$j=1$ bis $n$}
		\State Finde die größte Quote $q_{jk}$ bzgl. Ereignis $k$ unter allen Buchmachern $j \in J$
	\EndFor
	\State Setze $\omega_k \gets \max{q_{jk}}$
\EndFor
\If{$\sum_{k}^{}{\omega_k} < 1$}
	\State \textbf{return} $(j,k)$ für alle Ausgänge $k$
	\Else
		\State \textbf{return} $-1$
\EndIf
\EndProcedure
\end{algorithmic}
\end{algorithm}

Die Vorgehensweise von Algorithmus~\ref{alg:arbitrage} sei hier skizziert. Als Eingabe dienen die Quoten $ q_{jk} $ eines Fußballspiels mit zugehörigen Buchmachern. Die Hauptaufgabe des Algorithmus besteht aus einer Doppelschleife (Schritt 2 u. 3). Für jedes Ereignis $ k=H,U,A $ werden die Quoten von allen Buchmachern mit einander verglichen und die Größte gespeichert (Schritt 4\&6). In Schritt 8 wird die Existenzbedingung aus Satz~\ref{satz_arbitrage_erwartung} überprüft. Ist diese erfüllt, gibt der Algorithmus mit $ (j,k) $ den Index $ j $ des Buchmachers mit dem zugehörigen Ereignis $ k $ aus (Schritt 9). Ist die Existenzbedingung nicht erfüllt existiert für dieses Spiel keine Arbitrage und es wird ein Fehlercode (hier: -1) zurückgegeben. Es ist offensichtlich, dass die Erfolgswahrscheinlichkeit des Algorithmus mit zunehmender Menge der Buchmacher ebenfalls zunimmt. Insbesondere lässt sich die Bedingung in Schritt 8 noch verschärfen. Wählt man als Schwellenwert statt 1 beispielsweise 0.9804\footnote{Dieser Wert ergibt sich aus einer Einfachen Rechnung: $ EG^W = \tfrac{1}{T_s} - 1 > 0.02 \Leftrightarrow T_s < 0.9804 $} werden nur Spiele mit einer Mindestarbitrage von 2\% zurückgegeben.

\begin{table}
\centering
	\begin{tabular}{lcccccr}
	\toprule
	\textbf{Liga} & \textbf{Heim} & \textbf{Gast} & $ q_H $ & $ q_U $ & $ q_A $ & \textbf{Arb.} \\
	\midrule
	\textit{1. Bundesliga} & Hannover 96 & Herta BSC & 2.44 & 3.60 & 2.75 & 4.79\% \\
	\textit{Primera Division} & FC Sevilla & Rayo Vallecano & 1.60 & 4.92 & 7.50 & 3.99\% \\
	\textit{1. Bundesliga} & Braunschweig & Borussia Dortmund & 12.00 & 6.01 & 1.40 & 3.73\% \\
	\textit{1. Bundesliga} & Mönchengladbach & Bayern München & 7.69 & 4.75 & 1.60 & 3.57\% \\
	\textit{Primera Division} & Almeria & Levante & 2.10 & 3.65 & 4.59 & 3.30\% \\
	\textit{Premier League} & Southampton & Sunderland & 1.85 & 4.01 & 5.57 & 3.15\% \\
	\textit{1. Bundesliga} & Schalke & Hamburg SV & 1.60 & 4.86 & 7.20 & 3.13\% \\
	\textit{2. Bundesliga} & Kaiserslautern & Bielefeld & 1.45 & 5.19 & 11.39 & 3.08\% \\
	\textit{1. Bundesliga} & Hannover 96 & Augsburg & 2.65 & 3.50 & 3.25 & 3.01\% \\
	\textit{1. Bundesliga} & Schalke & Frankfurt & 1.65 & 4.69 & 6.54 & 2.86\% \\
	\textit{1. Bundesliga} & Augsburg & Schalke & 2.60 & 3.56 & 3.25 & 2.75\% \\
	\textit{1. Bundesliga} & Hamburg SV & Bayern München & 7.68 & 5.66 & 1.50 & 2.72\% \\
	\textit{Seria A} & Bologna & Roma & 8.42 & 4.35 & 1.60 & 2.71\% \\
	\textit{Primera Division} & Espanol & Almeria & 1.77 & 4.18 & 5.89 & 2.67\% \\
	\textit{Premier League} & Swansea & West Bromwich & 2.10 & 3.60 & 4.54 & 2.64\% \\
	\textit{1. Bundesliga} & Stuttgart & Bayern München & 12.45 & 6.50 & 1.35 & 2.57\% \\
	\toprule
	\end{tabular}
	
\caption[Ergebnisse der Arbitrageanalyse Saison 2013/14]{Ergebnisse der Arbitrageanalyse für die Saison 2013/14. Es wurden nur Ergebnisse dargestellt mit einer Arbitrage von mehr als 2.50\% \\ Quelle: Eigene Darstellung u. Berechnung}
\label{tab:arbitrageErgebnisse}
\end{table}


Dieser Algorithmus wurde auf den Datensatz der Saison 2013/14 angewendet. Die Ergebnisse sind in Tabelle~\ref{tab:arbitrageErgebnisse} dargestellt. Von 2124 untersuchten Spielen existierten bei 300 die Möglichkeit für Arbitrage. Das entspricht einem Anteil von 14.12\% des Datensatzes. Dieses Ergebnis wurde mit einem Schwellenwert von mindestens 0.01\% Arbitrage erzielt. Ab einer Mindestarbitrage von 1\% verringert sich die Anzahl der Spiele auf 93 von 2124. Ab 2\% Mindestarbitrage existierten nur noch bei 1 von 100 Spielen die Möglichkeit auf Arbitrage. Die durchschnittliche Arbitrage beträgt 0.83\%. Der maximal mögliche Arbitragegewinn beläuft sich auf 4.79\% im Spiel Hannover 96 gegen Herta BSC im Rahmen der 1. Bundesliga. In Tabelle~\ref{tab:arbitrageErgebnisse} sind die 16 Spiele mit einer Mindestarbitrage von mehr als 2.50\% gelistet. Auffällig ist, dass in jeder untersuchten Liga Arbitragemöglichkeiten zu finden sind. Besonders überraschend ist jedoch die Tatsache, dass der Großteil der Spiele mit den höchsten Arbitragegewinnen aus der 1. Bundesliga stammen. Dieser Umstand könnte dahingehend interpretiert werden, dass es sich bei der 1. Bundesliga um ein besonders volatilen und ''gern gespielten'' Wettmarkt handelt. In Kapitel 3 ist die Möglichkeit Spieler zu ''ködern'' angesprochen worden. Buchmacher können absichtlich Quoten mit Arbitragemöglichkeit anbieten um so Spieler dazu zu verleiten diese Quoten zu kaufen. Zusätzlich lässt sich kein abweichendes Verhalten der Buchmacher untereinander feststellen. Im Datensatz und sogar in den 16 Spielen mit den größten Arbitrageerträgen lässt sich jeder der untersuchten Buchmacher wiederfinden.

Diese Ergebnisse unterscheiden sich mit denen benachbarter Literatur beträchtlich. \citet[S. 434]{vlastakis2009efficient} fanden in jedem 200. Spiel die Möglichkeit auf Arbitrage. Reduziert auf elektronische Buchmacher fanden sich lediglich in jedem tausendstem Spiel Arbitrage. In ihrer Untersuchung fanden sie sogar ein Spiel mit mehr als 200\% möglicher Arbitrage. Ihr berechneter Erwartungsgewinn von durchschnittlich 21.78\% ist zudem beträchtlich größer als der in dieser Arbeit gefundene (hier im Mittel 0.83\%). Ursache könnte die Hinzunahme eines staatlichen Anbieters sein, welcher in der Untersuchung von \citeauthor{vlastakis2009efficient} Quoten anbot, welche nicht im Einklang mit den Quoten der Konkurrenz standen. Außerdem ist der Vergleich der Arbitrageergebnisse mit denen von \citeauthor{vlastakis2009efficient} schwierig, denn diese Arbeit untersuchte die Möglichkeit für Arbitrage auf Grundlage einer größeren Menge von Buchmachern, was zu einer höheren Wahrscheinlichkeit für die Existenz von Arbitrage führt. Die hier gefundenen Ergebnisse decken sich in Teilen mit denen von \citet[S. 361-363]{barth2012oekonomie}. Dessen durchschnittlicher Arbitragegewinn betrug 2.5\%. In nur 10 von 38 Begegnungen mit Arbitragemöglichkeit betrug diese mehr als 3\%. Obwohl sich die Arbitrageergebnisse in Umfang und Höhe unterscheiden bleiben die Gründe für das Vorhandensein von Arbitrage laut \citet[434]{vlastakis2009efficient} immer dieselben: Diskrepanzen in der Preissetzung, Informationsunterschiede und Unterschiede bezüglich des Wettklientels (vgl. risikominimale Wettquoten, Kapitel 3).

Laut Definition der \textit{schwachen Markteffizienz} muss die Möglichkeit für Arbitrage in einem Markt vollkommen ausgeschlossen sein. In dieser Hinsicht sind die Ergebnisse der Arbitrageanalyse für die Saison nicht im Einklang mit der Effizienzmarkthypothese. Jedoch lässt sich auf Grund des geringen Aufkommens von bedeutenden Arbitragemöglichkeiten (nur 5\% der Arbitrage-Spiele lieferten mehr als 2.50\% Arbitrage) die Definition der schwachen Markteffizienz als bestätigt interpretieren. Obwohl Arbitrage laut Definition risikoloser Gewinn bedeutet, muss im Bereich der Sportwetten eine differenzierte Betrachtung erfolgen. Um Arbitrage voll auszuschöpfen zu können muss ein Spieler eine Vielzahl von Konten bei verschiedenen Buchmachern eröffnen und jedes Konto mit Startkapital ausstatten. Nimmt man die geringe Durchschnittshöhe der Arbitrage hinzu muss das verwendete Startkapitel extrem hoch gewählt werden, um nennenswerte Rendite zu erwirtschaften. Hinzu kommt, dass ein Spieler zusätzlich einen immensen Zeitaufwand betreiben muss. Arbitragemöglichkeiten besitzen nur eine durchschnittliche Lebensdauer von 15 Minuten \citep{marshall2009quickly}, bevor diese durch Justieren der Quoten oder Wettabbrüche verschwinden. Das heißt die investierte Zeit muss als zusätzlicher Kostenfaktor einkalkuliert werden. Arbitrage-Software kann in dieser Hinsicht hilfreich sein, setzt jedoch ebenfalls eine Investition voraus und muss als Anschaffungskosten mit eingerechnet werden. Zu guter Letzt muss erwähnt werden, dass es zu einem enormen Verlust durch Wettabbrüche kommen kann. Kommt es zum Beispiel zu einem Fehler bei der Wettquotensetzung, steht es den Buchmachern nach geltendem Recht zu, ihr Wettangebot zurückzunehmen. Ein Spieler erhält zwar seinen Einsatz zurück, jedoch führt ein Wettabbruch dazu, dass mindestens ein Ereignis nicht gewettet wird und es so zu einem Verlust durch die bereits abgeschlossenen Wetten kommen kann. Das kann die Gewinne, die über lange Zeit erwirtschaftet wurden vernichten.

\section{Wettstrategien}
Dieser Abschnitt befasst sich mit der Suche nach lukrativen Wettstrategien. Die in dieser Arbeit untersuchten Wettstrategien leiten sich aus der Definition der Drei-Weg-Ergebnis-Wette ab. Sie lauten: Alle Ausgänge, nur Heimsiege, Unentschieden oder Auswärtssiege, nur Favoriten oder Außenseiter zu wetten. Damit der Markt der \textit{schwachen Markteffizienz} im Rahmen der Effizienzmarkthypothese genügt, dürfen Wettstrategien, welche sich aus Informationen (hier: Quoten) vergangener Saisons ableiten, nicht existieren. Für die Berechnung der Favoriten- beziehungsweise Außenseiterposition wurden die Quoten nur gewettet, wenn die implizite Wahrscheinlichkeit größer als 0.50 beziehungsweise kleiner als 0.20 gegeben war. Um eine messbare Größe für den Erfolg der Wettstrategie zu erhalten wurde der \textit{Return on Investment} (RoI) gewählt. Bezogen auf den Wettmarkt und die hier durchgeführte Analyse berechnet sich dieser als: \[ RoI_{strat} = \frac{Auszahlung^{strat} - Einsatz^{+} - Einsatz^{-}}{Einsatz^{ges}} \] Wobei mit $ Auszahlung^{strat} $ die Auszahlung durch die gespielten Wettquoten der zugehörigen Strategie und mit $ Einsatz^{+} $ sowie $ Einsatz^{-} $ die Einsätze der gewonnenen beziehungsweise verlorenen Wetten bezeichnet wird. Als $ Einsatz^{ges} $ ist der Gesamteinsatz bezeichnet.

\begin{table}
\centering
	\begin{tabular}{lcccccc}
	\toprule
	\textbf{Anbieter} & \textbf{Alle} & \textbf{H} & \textbf{U} & \textbf{A} & \textbf{F} & \textbf{L} \\
	\midrule
	\textit{Buchmacher B365} & -0.0747 & -0.0388 & -0.1227 & -0.0624 & -0.0227 & -0.1160 \\
	\textit{Buchmacher BW} & -0.0867 & -0.0500 & -0.1268 & -0.0834 & -0.0355 & -0.1641 \\
	\textit{Buchmacher IW} & -0.1036 & -0.0635 & -0.1442 & -0.1031 & -0.0308 & -0.1964 \\
	\textit{Buchmacher LB} & -0.0871 & -0.0475 & -0.1309 & -0.0830 & -0.0181 & -0.1461 \\
	\textit{Buchmacher PS} & -0.0336 & -0.0052 & -0.0884 & -0.0070 & -0.0057 & -0.0176 \\
	\textit{Buchmacher WH} & -0.0508 & -0.0136 & -0.1078 & -0.0311 & +0.0052 & -0.0498 \\
	\textit{Buchmacher SJ} & -0.0863 & -0.0458 & -0.1372 & -0.0760 & -0.0160 & -0.1327 \\
	\textit{Buchmacher VC} & -0.0481 & -0.0149 & -0.1024 & -0.0269 & -0.0077 & -0.0805 \\
	\toprule
	\end{tabular}

\caption[RoI ausgewählter Wettstrategien in der Saison 2013/14]{Der \textit{Return on Investment} ausgewählter Wettstrategien in der Saison 2013/14 für alle untersuchten Buchmacher. Die angewendeten Wettstrategien: Alle Wettausgänge (Alle), Nur Heimsieg (H), Nur Unentschieden (U), Nur Auswärtssieg (A), Nur Favoriten (F), Nur Außenseiter (L)  \\ Quelle: Eigene Darstellung u. Berechnung}
\label{tab:strat:alleBuchmacher}
\end{table}

Die Strategien wurden für den Datensatz der Saison 2013/14 durchgeführt. Die Ergebnisse sind in Tabelle~\ref{tab:strat:alleBuchmacher} aufgelistet. Es ist auffällig, dass fast jede Strategie zu einem negativen RoI führte. Einzig die Strategie nur Favoriten zu wetten, das heißt die Wettquoten müssen eine implizite Wahrscheinlichkeit von mehr als 50\% aufweisen, lieferte einem Spieler bei dem Buchmacher \textit{William Hill} einen RoI mit positivem Vorzeichen in Höhe von 0.52\%. Die ''schlechteste'' Strategie stellt das Wetten auf den Spielausgang Unentschieden in jeder Partie dar. Der Verlust reicht von -8.84\% bis -14.42\%. Wettstrategien mit der Prämisse nur einen Spielausgang (H),(U) oder (A) zu wetten liefern alle samt einen negativen RoI. Diese Tatsache ist wenig überraschend, denn die Ergebnisse des letzten Abschnitts haben gezeigt, dass alle Buchmacher den Ausgang eines Fußballspiels sehr exakt einschätzen können. Damit handelt es sich um markteffiziente Wettquoten und der Buchmacher erwirtschaftet im Mittel seine Gewinnmarge. Aus diesem Grund nähern sich die zugehörigen Werte $ RoI_H $ und $ RoI_A $ den Gewinnmargen der Buchmacher (vgl. Tabelle~\ref{tab:statistik:gewinnmarge}). Es ist außerdem ein deutlich größerer RoI für die Wettstrategie ''Nur Favoriten'' zu erkennen. Dieser reicht von -3.55\% bis +0.05\%. Dass die Favoriten-Wette eine lukrativere Strategie darstellt als die Anderen, ist charakteristisch für die Existenz eines \textit{Favourite-longshot bias}. Auf Gründe, weshalb Buchmacher einen FLB realisieren wurden am Ende von Kapitel 3 bereits erläutert.

\begin{table}[h]
\centering
	\begin{tabular}{lcr}
	\toprule
	\textbf{Strategie} & \textbf{Spiele} & \textbf{\textit{RoI}} \\
	\midrule
	Alle Ausgänge & 6372 & -0.0213 \\
	Nur Heimsiege & 2124 & +0.0083 \\
	Nur Unentschieden & 2124 & -0.0815 \\
	Nur Auswärtssiege & 2124 & +0.0092 \\
	Favoriten  & 1015 & +0.0049 \\
	Außenseiter  & 1137 & +0.0004\\
%	Favoriten B & 1080 & -0.0032 \\
%	Außenseiter B & 1201 & -0.0169 \\
	\toprule
	\end{tabular}
	
\caption[RoI der Wettstrategien über alle Buchmacher Saison 2013/14]{Der \textit{Return on Investment} ausgewählter Wettstrategien über alle Spiele und alle Buchmacher. \\ Quelle: Eigene Darstellung u. Berechnung}
\label{tab:wettstratOverAll}
\end{table}

Im nächsten Schritt geschieht die Betrachtung der Wettstrategien unter dem Blickwinkel der Gewinnmaximierung des Spielers. Ein Spieler mit dem Ziel der Gewinnmaximierung wird jeweils die größte Quote über alle Buchmacher bezüglich eines Fußballspiels wählen. Die Ergebnisse dieser Analyse sind in Tabelle~\ref{tab:wettstratOverAll} dargestellt. Die erhaltenen Werte sind durchaus überraschend. Die Strategie nur Favoriten-Wetten zu spielen liefert abermals einen positiven RoI, was die Existenz des FLB für den Wettmarkt in der Saison 2013/14 bestätigt. Zudem lassen sich positive RoIs für die Strategien \textit{Nur Heimsiege} (0.83\%), \textit{Nur Auswärtssiege} (0.92\%) sowie \textit{Außenseiter} (0.04\%) herausstellen. Der positive RoI für die Wettstrategie \textit{Nur Heimsiege} bestätigt die Existenz des schon oft empirisch belegten Heimvorteils (bswp. \citet{clarke1995home}) und wird in der englischsprachigen Literatur als \textit{Home bias} (dt. Heim-Verzerrung) bezeichnet. Dessen ungeachtet fallen die anderen Strategien mit positiven RoIs aus dem Raster und sind daher erstaunliche Funde. Es sei erwähnt, dass die Strategie nur Favoriten-Wetten vor dem Hintergrund der Gewinnmaximierung zu spielen nicht zwangsläufig zu einem größeren Gewinn führt, wenn der Spieler die größten Wettquoten aller Buchmacher spielt. Man würde vermuten, dass der RoI in diesem Falle den der selben Strategie, jedoch beschränkt auf den Buchmacher \textit{William Hill}, übertrifft (vgl. Tabelle~\ref{tab:strat:alleBuchmacher}, 0.05\%). Allerdings sei abermals auf den Zusammenhang zwischen Wettquote und impliziter Wahrscheinlichkeit verwiesen. Je größer die Wettquote, desto kleiner die implizite Wahrscheinlichkeit. Auf Grund der abweichenden Gewinnmargen (vgl. Tabelle~\ref{tab:statistik:gewinnmarge}) können so mit der Optimierung über alle Buchmacher bei der Strategie nur Favoriten-Wetten zu spielen ganz und gar verschiedene Wetten gespielt werden. Die Analyse der Wettstrategien über alle Buchmacher bezog sich mit der Vorgabe der Gewinnmaximierung nur auf die Wahl der größten Wettquoten und nicht auf die größten impliziten Wahrscheinlichkeiten.

Die gefundenen Ergebnisse der Effizienzanalyse der Saison 2013/14 passen von der Größenordnung zu denen der benachbarten Literatur. \citet{kossmeier2008efficiency} stellten in ihrer Untersuchung der Saison 2000/01 ebenfalls einen FLB fest. Sie bemerkten außerdem, dass die Existenz des FLB durchaus erstaunlich ist, da es sich bei dem elektronischen Wettmarkt um einen sehr transparenten und konkurrenzbetonten Markt handelt (vgl. Tabelle~\ref{tab:statistik:gewinnmarge} u. Interpretation des Takes). \citeauthor{kossmeier2008efficiency} interpretierten ihre Ergebnisse dahingehend so, dass der Wettmarkt die Effizienzmarkthypothese nach Fama nicht erfüllt. In ihrer Untersuchung der Saison 2002/03/04 durch \citet{vlastakis2009efficient} fanden die Autoren ebenfalls einen FLB und kamen zu dem Schluss, dass der Wettmarkt der \textit{schwachen Markteffizienz} nicht genügt. Sie stützen ihre Konklusion jedoch insbesondere auf Ergebnisse der Arbitrageanalyse, welche stark von den hier nachgewiesen Arbitragemöglichkeiten abwichen (vgl. Tabelle~\ref{tab:arbitrageErgebnisse}). Nach Auswertung der Wettstrategien schließt sich diese Arbeit den Autoren der anderen Werke an und stellt fest, dass die Definition der \textit{schwachen Markteffizienz} durch eindeutige Existenz des FLB und lukrativer Wettstrategien verletzt ist.














