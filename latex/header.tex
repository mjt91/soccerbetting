\documentclass[paper=a4,toc=bibliography,chapterprefix,parskip=true]{scrreprt}

%%kindl-header
%\documentclass[10pt]{report}
%\usepackage[paperwidth=13cm, paperheight=21cm, top=1.5cm, left=1cm,
%right=1cm, bottom=0.5cm]{geometry}
% -------------------------------------------------------------------
%%% Laden elementarer Pakete
%
% Deutsche Schriftpakete
\usepackage[utf8]{inputenc}              % alternativ: 'utf8' oder 'latin9' statt ansinew
%\usepackage[ansinew]{inputenc}              % alternativ: 'utf8' oder 'latin9' statt ansinew
\usepackage[round]{natbib} 				% Nat Bib für Zitierweise nach DIN 1505
\usepackage[TS1,T1]{fontenc}
\usepackage{lmodern,textcomp}
\usepackage[english,ngerman]{babel}
%
% Mathematische Paktete
\usepackage{amsmath,amssymb,bm,bbm}         % Formelsetzung und mathematischen Symbole
\usepackage[amsmath,thmmarks]{ntheorem}     % Theorem-Umgebungen, alternativ: 'amsthm'

% Algorithmus
\usepackage{algorithm}
\usepackage{algorithmicx}
\usepackage{algpseudocode}
\algrenewcommand\algorithmicrequire{\textbf{Input:}}
\algrenewcommand\algorithmicensure{\textbf{Output:}}
%
% Grafik-Pakete einbinden
\usepackage{graphicx,psfrag}                % Basis-Pakete zum Laden von Bildern (jpg?)
\usepackage{float}
\usepackage{color}                          % erweitertes Farb-Paket, alternativ: 'xcolor'
%\usepackage{pstricks,pst-plot}              % weiteres Paket zur Erstellung von LaTeX-Grafiken
%
% erweiterte Tabellen
\usepackage{array}                          % Basis-Paket
\usepackage{booktabs}                       % 'schöne' Tabellen
\usepackage{tabularx}                       % Tabellen mit dynamischer Spaltenbreite
\usepackage{longtable}                      % Tabellen mit möglichem Seitenumbruch
\usepackage{multirow}                       % mehrzeilige Zellen
\usepackage{pbox}


%Tabellen-Prämbel:
\usepackage[center]{caption}
\addto\captionsngerman{\renewcommand{\tablename}{\small{Tab.}}}
\captionsetup{tablewithin = section}
\captionsetup{font=small, labelfont=bf}



%
% weitere
%\usepackage{bibgerm}
\usepackage{verbatim, listings}             % Darstellung von Quellcode
%\usepackage{cite}
\lstset{language=TeX,basicstyle=\footnotesize,frame=single,breaklines=true}
\usepackage[format=plain,indention=.5cm]{caption} % für selbsdefinierte captions
\usepackage{stmaryrd}                       % Blitzsymbol bei Widerspruch
%\usepackage[square]{natbib}                 % naturwissenschaftliche Zitierweise
%
% Paket für interne Links
\usepackage[%
	breaklinks=true    % Links »überstehen« Zeilenumbruch
  	,colorlinks        % Links erhalten Farben statt Kästen
  	,linkcolor=black   % beeinflusst Inhaltsverzeichnis und Seitenzahlen
  	,urlcolor=black    % Farbe für URLs
    ,citecolor=black
  	,bookmarks         % Erzeugung von Bookmarks für PDF-Viewer
  	,bookmarksnumbered % Nummerierung der Bookmarks
]{hyperref}
\usepackage{breakurl}
%
% -------------------------------------------------------------------


% -------------------------------------------------------------------
%%% Seitenstil
%
\usepackage{scrpage2}                       % Kopf- und Fußzeilenformatierung
\usepackage[onehalfspacing]{setspace}       % Zeilenabstand = 1,5
\recalctypearea
\pagestyle{scrheadings}
\automark[section]{chapter}
\addtokomafont{sectioning}{\rmfamily}
%
% -------------------------------------------------------------------


% -------------------------------------------------------------------
%%% Einstellungen und Formatierung der Theorem-Umgebung
%
% Stil der Definition - Umgebung
\theoremstyle{break}
\theoremheaderfont{\sffamily\bfseries}
\theorembodyfont{\upshape}
\theoremsymbol{}
\newtheorem{definition}{Definition}[chapter]
\newtheorem{satz}[definition]{Satz}
\newtheorem{resultat}[definition]{Resultat}
\newtheorem{lemma}[definition]{Lemma}
\newtheorem{folgerung}[definition]{Folgerung}
\newtheorem{korollar}[definition]{Korollar}
%
\theorembodyfont{\rmfamily}
\newtheorem{bemerkung}[definition]{Bemerkung}
\newtheorem{beispiel}[definition]{Beispiel}
%
% Stil der Beweis - Umgebung
\theoremstyle{nonumberplain}
\theoremsymbol{\ensuremath{\Box}}
\newtheorem{beweis}{Beweis:}
%
% ------------------------------------------------------------------- 